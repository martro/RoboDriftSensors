\documentclass[a4paper,10pt]{article}
%\documentclass[a4paper,10pt]{scrartcl}

\usepackage{polski}
\usepackage[T1]{fontenc}
\usepackage[utf8]{inputenc}

\title{Programowanie w C}
\author{Jakub Trzyna}
\date{04.01.2014}

\pdfinfo{
  /Title    (readme.pdf)
  /Author   (Jakub Trzyna)
}

\begin{document}
\maketitle

\begin{abstract}
 Dokument przedstawia opis działania programu napisanego jako trzeci projekt na zajęcia z programowania w języku C.
 Program umożliwia dodawanie, usuwanie, edycję oraz wyświetlanie bazy gitar.
\end{abstract}

\section{Dodawanie/Usuwanie/Edytowanie gitary}
\begin{enumerate}
\item Dodawać nowe gitary można poprzez użycie opcji:
\begin{itemize}
\item>>11<< załadowanie wskazanych/wszystkich elementów z pliku binarnego o rozszezrzeniu .jt. Rozszerzenie jest automatycznie dodawane przez program, wiec użytkownik musi się martwić tylko o poprawną nazwę pliku.
\item>>12<< utworzenie nowego elementu o parametrach wybranych przez użytkownika
\item>>13<< wygenerowanie dowolnej, wskazanej ilości losowych gitar.
Funkcja losująca posiada w bazie kilka marek, typów budów i rodzajów z 
których losuje pojednyncze elementy przyporządkowując je do tworzonej gitary.
\end{itemize}

\item Aby usunąć konkretną gitarę z bazy należy wybrać opcję >>14<< w menu głównym, a następnie podać numer gitary którą chcemy usunąć.
\item Edycja parametrów gitary następuje po wybraniu opcji >>15<< z menu głównego. Po wpisaniu numeru gitary, program poprosi o podanie kolejnych parametrów, jednocześnie pokazując w nawiasie dotychczasową wartość.
\end{enumerate}

\section{Wyświetlanie listy gitar}
 Funkcja >>2<< umożliwia wyświetlenie wszystkich gitar wraz z ich parametrami które znajdują sie w bazie. 

\end{document}
